\documentclass[11pt]{article}

\usepackage[T1]{fontenc}
\usepackage[utf8]{inputenc}
\usepackage{sectsty}
\usepackage{graphicx}
\usepackage{amsmath,amssymb,amsfonts}
\usepackage{algorithmic}
\usepackage{graphicx}
\usepackage{textcomp}
\usepackage{xcolor}
\usepackage{listings}
\usepackage{xpatch}
\usepackage{realboxes}
\usepackage{hyperref}
\usepackage[portuguese]{babel}
\usepackage{biblatex} %Imports biblatex package
\addbibresource{bibliography.bib}
\usepackage{subfig}
\def\BibTeX{{\rm B\kern-.05em{\sc i\kern-.025em b}\kern-.08em
    T\kern-.1667em\lower.7ex\hbox{E}\kern-.125emX}}

\definecolor{mygray}{rgb}{0.88,0.88,0.88}
\lstset{
  basicstyle=\ttfamily,
  backgroundcolor=\color{mygray},
  texcl=false,
  language=C,
  numbers=left,
  numberstyle=\tiny\color{gray},
  keywordstyle=\color{blue},
  commentstyle=\color{green}\ttfamily,
  stringstyle=\color{mauve},
}
\makeatletter
\xpretocmd\lstinline
  {%
   \bgroup\fboxsep=1pt
   \Colorbox{mygray}\bgroup\kern-\fboxsep\vphantom{\ttfamily\char`\\y}%
   \appto\lst@DeInit{\kern-\fboxsep\egroup\egroup}%
  }{}{}
\makeatother

% Define \code
\newcommand{\code}{\lstinline[mathescape=true]}

% Define \floor
\newcommand{\floor}[1]{\lfloor #1 \rfloor}

% Margins
\topmargin=-0.45in
\evensidemargin=0in
\oddsidemargin=0in
\textwidth=6.5in
\textheight=9.0in
\headsep=0.25in

\title{ Programação Concorrente 2021/1 \\
\Large{ Trabalho Final }}
\author{ Leonardo Alves Riether \\ 190032413 }
\date{\today}

\begin{document}
\maketitle
\pagebreak

% Optional TOC
% \tableofcontents
% \pagebreak

%--Paper--

\section{Introdução}

% TODO:
Lorem Impsum

\section{Problema Proposto}
    O contexto do problema proposto para o trabalho é o de uma maratona de programação, estilo Maratona
    SBC de Programação\cite{maratonasbc} ou ICPC\cite{icpc}.

    Em uma maratona, vários times de três competidores possuem uma prova com diversos problemas para
    resolver.  À medida que os programadores encontram as soluções, eles precisam escrever um programa
    para cada questão que resolva o problema e passe em todos os casos de teste. Ao terminar de escrever
    um código, o competidor a envia ao juíz automático, que demora um tempo considerável para gerar o
    veredito (aceita ou rejeita a solução).

    Cada time possui apenas um computador, por isso no máximo uma pessoa do time pode utilizá-lo em um
    dado instante de tempo\footnote{Nas últimas competições, virtuais, esse não foi o caso, porém em
    competições presenciais tradicionais isso é verdade.}. Para deixar a utilização do computador mais
    justa, os times seguem a seguinte estratégia: se membro do time encontrar a solução para uma questão,
    mas outra pessoa já está no computador, ele escreve seu nome em uma folha de papel, que servirá de
    fila. Assim que a pessoa de posse do computador acaba de escrever o código e o envia ao juíz, ela
    verifica a folha de papel e chama o primeiro da fila, caso ela não esteja vazia. Como um competidor
    pode solucionar várias questões antes de ter acesso ao computador, ele pode colocar seu nome várias
    vezes na folha, uma vez para cada questão solucionada.

    Outra ação que os competidores podem tomar durante a prova é ir para a sala de coffee break. Visto
    que não há muito espaço, um número pequeno de pessoas pode ficar nela simultaneamente. Se um
    indivíduo resolver ir para o coffee break enquanto a sala está cheia, ele fica esperando na porta até
    que consiga entrar.

    Para que o começo da prova não seja injusto, cada competidor aguarda até que todos estejam prontos
    para começar a prova.

    A fim de aumentar o grau de concorrência, é permitido que os times possuam mais de três membros.

    Por fim, também é proposta a implementação do juíz automático, que possui uma fila de submissões a
    serem julgadas. Todos os times enviam seus programas para o mesmo juíz, colocando-as no final da
    fila. O juíz, por sua vez, julga as submissões assim que possível, retirando da fila na ordem FIFO
    (first in, first out), e pode dar os vereditos AC (Accepted), TLE (Time Limit Exceeded) ou WA (Wrong
    Answer).

\section{Estrutura do Código}
	A implementação em C foi compilada com o seguinte comando shell, incluído no código fonte no arquivo
	\code{run}:

	\begin{lstlisting}[language=bash, caption=run]
gcc -DCOLOR -pthread -g \
    -Wall -pedantic -Wextra -Wshadow -fsanitize=address,undefined \
    -o out main.c
	\end{lstlisting}

	É possível omitir a flag \code{-DCOLOR}, caso o terminal não tenha suporte aos códigos de escape de
	cor ANSI\cite{ansi_color}, mas a flag é recomendada, para deixar as mensagens de saída mais fáceis de
	se identificar visualmente. Além disso, foi utilizado GCC versão 7.5.0.

	O código foi dividido em módulos, mas a maior parte da implementação dos comportamentos descritos no
	problema proposto está no arquivo \code{main.c}.

\section{Solução}

\subsection{Comportamento dos Competidores}
	No início do programa, são criadas $N$ threads, uma para cada competidor. Há várias variáveis
	associadas ao comportamento dos competidores, % TODO TODO TODO TODO TODO TODO TODO TODO TODO TODO TODO TODO TODO TODO TODO
	Há vários requerimentos que dizem respeito ao comportamento dos competidores.

	\subsubsection{Acesso Exclusivo ao Computador}
	
	\subsubsection{}

	\subsubsection{Coffee Break}
	O alarme que indica que um competidor conseguiu resolver uma questão foi reaproveitado para a
	implementação do coffee break, para simplificar um pouco o código. Quando um evento do tipo
	\code{Alarme} é recebido, há uma probabilidade de 20\% do competidor, em vez de resolver um problema,
	resolver ir para a sala do coffee break. Quando isso acontece, é chamada a função
	\code{entrar_no_coffee_break}, que utiliza o semáforo global \code{coffee_break} para permitir que no
	máximo \code{MAX_COFFEE} pessoas entrem na sala.

	Dentro dessa função, as threads dormem de 3 a 7 segundos, enquanto o competidor aproveita o coffee
	break, depois executam um \code{sem_post} para sair da sala e retornam da função.


\subsection{Comportamento do Juíz Automático}



\section{Conclusão}

% TODO:

\medskip
\printbibliography

\end{document}
